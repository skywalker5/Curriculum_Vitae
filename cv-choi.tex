%%%%%%%%%%%%%%%%%%%%%%%%%%%%%%%%%%%%%%%%%
% Medium Length Professional CV
% LaTeX Template
% Version 2.0 (8/5/13)
%
% This template has been downloaded from:
% http://www.LaTeXTemplates.com
%
% Original author:
% Trey Hunner (http://www.treyhunner.com/)
%
% Important note:
% This template requires the resume.cls file to be in the same directory as the
% .tex file. The resume.cls file provides the resume style used for structuring the
% document.
%
%%%%%%%%%%%%%%%%%%%%%%%%%%%%%%%%%%%%%%%%%

%----------------------------------------------------------------------------------------
%	PACKAGES AND OTHER DOCUMENT CONFIGURATIONS
%----------------------------------------------------------------------------------------

\documentclass{resume} % Use the custom resume.cls style
\usepackage[pdftex]{hyperref}
\usepackage{fancyhdr}
\usepackage[mmddyyyy]{datetime}
\usepackage[dvipsnames]{xcolor}

\usepackage[left=0.75in,top=1.2in,right=0.75in,bottom=1.2in]{geometry} % Document margins



\newcommand{\tab}[1]{\hspace{.2667\textwidth}\rlap{#1}}
\newcommand{\itab}[1]{\hspace{0em}\rlap{#1}}
\name{Dongjin Choi} % Your name
\address{Klaus Advanced Computing Building, 1305} % Your address
\address{266 Ferst Drive, Atlanta, GA 30332, USA} % Your secondary addess (optional)
\address{\href{https://www.cc.gatech.edu/~dchoi85/}{www.cc.gatech.edu/$\sim$dchoi85/} \\ \href{mailto:jin.choi@gatech.edu}{jin.choi@gatech.edu}} % Your phone number and email

\renewenvironment{rSection}[1]{
	\sectionskip
	\textcolor{Black}{\MakeUppercase{#1}}
	\sectionlineskip
	\hrule
	\begin{list}{}{
			\setlength{\leftmargin}{1.5em}
		}
		\item[]
	}{
	\end{list}
}


\urlstyle{same}
\pagestyle{plain} 
\pagestyle{fancy}
\fancyfoot[R]{Last updated: \today} %set date/time on right
\renewcommand{\headrulewidth}{0pt} % remove awful top bar
\begin{document}
	
	%----------------------------------------------------------------------------------------
	%	Interest SECTION
	%----------------------------------------------------------------------------------------
%	\pagestyle{headings}
%	\fancyhead{} %clear out default
	\begin{rSection}{Research Interest}
		Probabilistic Topic Model, Numerical Machine Learning\hfill
		\vspace{-2mm}
	\end{rSection}
	
	%----------------------------------------------------------------------------------------
	%	EDUCATION SECTION
	%----------------------------------------------------------------------------------------
	
	\vspace*{-2.0mm}
	\begin{rSection}{Education}
		\vspace{-1mm}		
		{\bf Georgia Institute of Technology}, Atlanta, GA \hfill { Aug 2018 - Present} 
		\\ Ph.D. in Computational Science and Engineering  
		\\ Advisor: Prof. Haesun Park\hfill
		
		\vspace*{-2.5mm}
		{\bf Seoul National University}, Seoul, Korea \hfill { Mar 2011 - Feb 2018} 
		\\ B.S. in Electrical and Computer Engineering \hfill
		\\ Minor in Computer Science \& Engineering \hfill
				
	\end{rSection}
\vspace*{-2.5mm}

%----------------------------------------------------------------------------------------
%	RESEARCH EXPERIENCE SECTION
%----------------------------------------------------------------------------------------

\vspace*{-2.0mm}
\begin{rSection}{Research Experience}
%		\vspace{1mm}	
\begin{rSubsection}{NAVER Search Engine Model}{Apr 2018 - Jul 2018}{Research Intern (Advisor: Jaegul Choi)}{NAVER Corp.}
	\item Proposed a novel algorithm for personalized search engine model
\end{rSubsection}	
\vspace*{-2.5mm}
\begin{rSubsection}{Data Mining Laboratory}{Aug 2016 - Jan 2018}{Research Intern (Advisor: Professor U Kang, Lee Sael)}{Seoul National University}
	\item Proposed a novel scalable CMTF algorithm using parallelization and caching computation results
	\begin{itemize}
		\vspace*{-2.5mm}
		\item Contributed as the first author for a paper submitted to \href{https://journals.plos.org/plosone/}{\textit{PLOS ONE}}
	\end{itemize}
	\item Applied network-regularized tensor factorization to a patient genetic mutation dataset
	\begin{itemize}
		\vspace*{-2.0mm}
		\item Contributed as the first author for a paper submitted to \href{https://www.computer.org/web/tcbb}{\textit{IEEE TCBB}}
	\end{itemize}
	\item Proposed a novel algorithm for sampling based dynamic tensor decomposition
	\begin{itemize}
		\vspace*{-2.0mm}
		\item Contributed as a co-author for a paper published by \href{http://journals.plos.org/plosone/}{\textit{PLOS ONE}}
		\vspace*{-2.0mm}
		\item Awarded as bronze prize for Humantech paper award @\em{\href{https://humantech.samsung.com/saitext/index.jsp}{Samsung}}\em
	\end{itemize}
	\item Proposed a novel system and algorithms to track SVD of multiple time series data
	\begin{itemize}
		\vspace*{-2.0mm}
		\item Contributed as a co-author for a paper published by \href{https://www.cikm2018.units.it/}{\textit{CIKM'18}}
	\end{itemize}
	\item Performed projects on building occupancy recognition and prediction for Intelligent Building Systems
	\begin{itemize}
		\vspace*{-2.0mm}
		\item Developed wireless sensor communication module using Arduino micro-controller boards
		\vspace*{-2.0mm}
		\item Developed a pedestrian simulator model
		\vspace*{-2.0mm}
		\item Implemented \em{ResNet}\em-based transfer learning network
	\end{itemize}
\end{rSubsection}
\vspace*{-2.0mm}

\begin{rSubsection}{Knowledge Discovery \& Database Laboratory}{Dec 2015 - Feb 2016}{Research Intern (Advisor: Professor Kyuseok Shim)}{Seoul National University}
	\item Implemented a previously proposed strategy on boosting subgraph isomorphism algorithms
	\item Found out useful vertex relationships in a graph and exploited them to boost up currently existing \textit{backtracking algorithms} for subgraph isomorphism
	\item Implemented distributed algorithms using Hadoop MapReduce
	%\vspace{1.3cm}
\end{rSubsection}

\end{rSection}


%----------------------------------------------------------------------------------------
%	Publications
%----------------------------------------------------------------------------------------

\begin{rSection}{Publications}
\begin{rSubsection}{}{}{}{}
	\vspace*{-2.5mm}
	
	\item \textbf{Dongjin Choi}, and Lee Sael, \textit{SNeCT: Integrative cancer data analysis via large scale network constrained Tucker decomposition}, \textit{IEEE TCBB}, 2019.
	
	\item Jun-gi Jang, \textbf{Dongjin Choi}, and U Kang, \textit{Fast and Memory Efficient Method for Time Ranged Singular Value Decomposition}, 27th ACM International Conference on Information and Knowledge Management (CIKM) 2018, Turin, Italy.
	
	\item Jungwoo Lee, \textbf{Dongjin Choi}, and Lee Sael, \textit{CTD: Fast, Accurate, and Interpretable Method for Static and Dynamic Tensor Decompositions}, PLOS ONE, 2018.
	
	\item Woojung Jin, \textbf{Dongjin Choi}, Youngjin Kim, and U Kang, \textit{Activity Prediction from Sensor Data using Convolutional Neural Networks and an Efficient Compression Method }, KIISE journal, 2018.
	
	\item \textbf{Dongjin Choi}, Jun-gi Jang, and U Kang, \textit{S3CMTF: Fast, Accurate, and Scalable Method for Incomplete Coupled Matrix-Tensor Factorization}, arXiv:1708.08640 [cs.NA], (submitted to \textit{PLOS ONE})
	
\end{rSubsection}

\end{rSection}
\vspace{-3mm}


%----------------------------------------------------------------------------------------
%	Patents
%----------------------------------------------------------------------------------------

\begin{rSection}{Patents}
	\begin{rSubsection}{}{}{}{}
		\vspace*{-2.5mm}
		\item U Kang, Jun-Gi Jang, \textbf{Dongjin Choi}, and Jinhong Jung, \textit{Apparatus and Method for Processing Data}, Korean Patent 10-2017-0159167, 2017.
		\item U Kang, \textbf{Dongjin Choi}, and Jun-gi Jang, \textit{Data Analysis Method and Apparatus for Sparse Data}, Korean Patent 10-2017-0158496, 2017.
	\end{rSubsection}
	
\end{rSection}
\vspace{-3mm}

%----------------------------------------------------------------------------------------
%	RESEARCH EXPERIENCE SECTION
%----------------------------------------------------------------------------------------

\begin{rSection}{Awards and Honors}
	\vspace*{-2.5mm}
	\begin{rSubsection}{}{}{}{}
		\item \textbf{Honorable Mention}, Humantech Paper Award, \em{\href{https://humantech.samsung.com/saitext/index.jsp}{Samsung}}\em \hfill Feb 2018
		
		\item \textbf{Bronze Prize}, Humantech Paper Award, top 6 in the CS division, \em{\href{https://humantech.samsung.com/saitext/index.jsp}{Samsung}}\em \hfill Feb 2017
		
		\item \textbf{National Science \& Technology Scholarship}, top 0.7\% in Korea, \href{http://www.kosaf.go.kr/}{KOSAF} \hfill 2011 - 2016
		
		\item \textbf{Kwon Oh-Hyun Alumni Scholarship}, additional 2,500\$/semester, \em Samsung \em \hfill 2015 - 2016
		
	\end{rSubsection}

\end{rSection}
%\vspace{-3mm}

%----------------------------------------------------------------------------------------
%	PROJECTS
%----------------------------------------------------------------------------------------

\begin{rSection}{Projects}
	\begin{rSubsection}{People flow recognition and prediction}{Sep 2017 - Jan 2018}{With Sovico, Samsung (Advisor: Professor U Kang)}{Seoul National University}
		\item Implemented a pedestrian simulator model
		\item Proposed isolated kernel CNN model for people flow recognition
		\item Proposed multi-scale skip connected and graph-structured RNN model for people flow prediction
	\end{rSubsection}	
	\begin{rSubsection}{Room occupancy detection for HVAC control}{Aug 2016 - Sep 2017}{With Smart Campus, Samsung (Advisor: Professor U Kang)}{Seoul National University}
		\item Developed IoT sensor kits using Arduino board
		\item Implemented server storage system with TCP communication via Wi-Fi
		\item Applied ResNet-based CNN network with transfer learning for real-time recognition of people count and activity
		\item Proposed RNN network for future-time prediction of people count and activity
	\end{rSubsection}	
%	\begin{rSubsection}{Deep learning course term project}{Mar 2017 - June 2017}{Term project for graduate deep learning course M1522.001600}{Seoul National University}
%		\item Human identification from image data
%		\item Crawled and collected human image data
%		\item Identify human characteristics (gender, ethnicity) from images, with high accuracy using transfer learning technique
%	\end{rSubsection}
	
\end{rSection}
%\vspace{1cm}

	%----------------------------------------------------------------------------------------
	%	SKILLS
	%----------------------------------------------------------------------------------------
	
%	\begin{rSection}{Skills}
%		%\vspace{-1mm}		
%		\begin{tabular}{ @{} >{\bfseries}l @{\hspace{6ex}} l }
%			Languages \& Libraries &  C++, MATLAB, Python \hfill (\textbf{Advanced}) \\
%			& Tensor Toolbox, Armadillo, JAMA\\
%			& Java, Perl, R, HTML \hfill (\textbf{Intermediate})\\
%			& Keras, TensorFlow\\
%		\end{tabular}
%		
%	\end{rSection}
%\vspace{-3mm}
	
	%----------------------------------------------------------------------------------------
	%	Reference
	%----------------------------------------------------------------------------------------
	
	\begin{rSection}{Reference}
	Available on request
				
	\end{rSection}
		
	
\end{document}
